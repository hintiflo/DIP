\documentclass{article}

\usepackage[utf8]{inputenc}
% \usepackage[latin1]{inputenc}
\usepackage[ngerman,naustrian]{babel}
\usepackage{lmodern}
\usepackage[T1]{fontenc}
\usepackage{ulem}
\usepackage{here}
\usepackage[pdftex]{graphicx}
\usepackage{amsthm}
\usepackage{gensymb}
\usepackage{fancyhdr}
\usepackage[left=20mm,top=25mm,bottom=25mm,right=20mm,headheight=15mm,headsep=10mm,footskip=10mm]{geometry}
\usepackage{longtable}
\usepackage{hhline}
\usepackage[table]{xcolor}
\usepackage{amsfonts}
\usepackage{amssymb}
\usepackage{amsmath}
\usepackage{mathcomp}
\usepackage{tabularx}
\usepackage{multicol}
\usepackage{graphicx}
\usepackage{listings}
\renewcommand{\familydefault}{\sfdefault}
		% \sffamily

\newcommand{\bild}[3]{
% \begin{figure}[H!]		
\begin{center}			\includegraphics[#3]{#1}			
% \caption{#2}
		\end{center}	
% \end{figure}
}

\begin{document}

%\pagestyle{empty}
\begin{titlepage}
	\begin{center}
		{\large{FH OÖ - Hagenberg \\ embedded systems design}\\\vspace*{4cm}}
		\small{HSC2-Übung}\\
		\textbf{SS 2021}\\\vspace*{2cm}
		\Huge{\textbf{Protokoll}}\\\vspace*{1cm}
		\huge{Übung\,1: CORDIC-Algorithums\\ \large{Berechnung von Winkelfunktionen auf einem FPGA via SystemC} } \vspace*{90mm}
		
		\small{Florian Berghuber S2010567001 \\
		Florian Hinterleitner S2010567014	\\
		}
	\end{center}
\end{titlepage}

% \tableofcontents
% \newpage
% \setcounter{page}{18}
% \setcounter{section}{3}
 
\section{Übungsaufgabe: Cordic-Algorithmus}
% screenshots
% Timing-tabelle

\subsection{Interface}
Eine register- und bitgenaue Implementierung des Interfaces zum Cordic soll eine nahtlose Anbindung des Algorithmus an den rest der Firmware erleichtern: \\
	% \bild{Interface}{Interface}{width=0.45\textwidth}
\subsection{Algorithmus}
Der Algorithmus selbst wurde entsprechend dem MATLAB-Prototypen in CPP umgesetzt.
Erweitert allerdings um das Handling der Status-Flags 'Start' und 'Ready'.

	% \bild{Algo}{Implementierung}{width=0.45\textwidth}
	% \lstinputlisting[language=C++]{../ue1/cordic.cpp}

\subsection{Testing}
	Alle Testcases im geforderten Intervall [0°, 90°] liefern korrekte Werte, verglichen mit Werten generiert aus der CPP-Math-Library. \\
	Die relativen Fehler für y im TC1 und für x im TC10 geben zwar alarmierende Werte aus, können jedoch als ungültig ignoriert werden. Dies liegt daran da jeweils durch sehr kleine Zahlen nahe der 0 dividiert wird, die berechneten Winkelfunktionen sind jedoch korrekt. \\
	Mit negativen Werten angeregt, liefert der Algorithmus falsche Ergebnisse, dieser Wertebereich war jedoch auch nicht spezifiziert, ebenso Eingaben oberhalb von 90°.
	% \bild{ConsoleOutput}{ConsoleOutput}{width=0.5\textwidth}

\end{document} 